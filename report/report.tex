%===================================================================================
% JORNADA CIENTÍFICA ESTUDIANTIL 2013 - MATCOM, UH
%===================================================================================
% Esta plantilla ha sido diseñada para ser usada en los artículos de la
% Jornada Científica Estudiantil, MatCom 2015.
%
% Por favor, siga las instrucciones de esta plantilla y rellene en las secciones
% correspondientes.
%
% NOTA: Necesitará el archivo 'jcematcom.sty' en la misma carpeta donde esté este
%       archivo para poder utilizar esta plantila.
%===================================================================================



%===================================================================================
% PREÁMBULO
%-----------------------------------------------------------------------------------
\documentclass[a4paper,10pt,twocolumn]{article}

%===================================================================================
% Paquetes
%-----------------------------------------------------------------------------------
\usepackage{amsmath}
\usepackage{amsfonts}
\usepackage{amssymb}
\usepackage{jcematcom}
\usepackage[utf8]{inputenc}
\usepackage{listings}
\usepackage[pdftex]{hyperref}
%-----------------------------------------------------------------------------------
% Configuración
%-----------------------------------------------------------------------------------
\hypersetup{colorlinks,%
	    citecolor=black,%
	    filecolor=black,%
	    linkcolor=black,%
	    urlcolor=blue}

%===================================================================================



%===================================================================================
% Presentacion
%-----------------------------------------------------------------------------------
% Título
%-----------------------------------------------------------------------------------
\title{Generación Automática de Configuraciones Visuales}

%-----------------------------------------------------------------------------------
% Autores
%-----------------------------------------------------------------------------------
\author{\\
\name Victor Manuel Cardentey Fundora \email \href{mailto:a.uno@lab.matcom.uh.cu}{a.uno@lab.matcom.uh.cu}
	\\ \addr Grupo C511 \AND
\name Karla Olivera Hernández \email \href{mailto:a.dos@lab.matcom.uh.cu}{a.dos@lab.matcom.uh.cu}
  \\ \addr Grupo C511 \AND
\name Amanda González Borrell \email \href{mailto:}{a.tres@lab.matcom.uh.cu}
  \\ \addr Grupo C511
  }

%-----------------------------------------------------------------------------------
% Tutores
%-----------------------------------------------------------------------------------
\tutors{\\
Lic. Daniel Alejandro Valdés Pérez \\
Lic. Ernesto Estevanell}

%-----------------------------------------------------------------------------------
% Headings
%-----------------------------------------------------------------------------------

%-----------------------------------------------------------------------------------
%===================================================================================



%===================================================================================
% DOCUMENTO
%-----------------------------------------------------------------------------------
\begin{document}

%-----------------------------------------------------------------------------------
% NO BORRAR ESTA LINEA!
%-----------------------------------------------------------------------------------
\twocolumn[
%-----------------------------------------------------------------------------------

\maketitle

%===================================================================================
% Resumen y Abstract
%-----------------------------------------------------------------------------------
\selectlanguage{spanish} % Para producir el documento en Español

%-----------------------------------------------------------------------------------
% Resumen en Español
%-----------------------------------------------------------------------------------
\begin{abstract}

	La generación automática de visualizaciones sobre un conjunto de
    datos se puede dividir en dos procesos: determinar una consulta de interés
    para el usuario y generar la configuración gráfica para visualizar los resultados de la
    consulta. En particular la selección de configuraciones gráficas es un problema que
    presenta dificultades para llegar a consenso entre expertos del dominio y los sistemas
    tradicionales que brindan solución a este problema utilizan enfoques basados en reglas.
    En años recientes se ha planteado la posibilidad de aplicar técnicas de \textit{Machine Learning}
    ampliamente utilizadas en sistemas de recomendación tradicionales a la recomendación de configuraciones
    gráficas. La propuesta de este trabajo consiste en utilizar y comparar distintos modelos
    de \textit{Machine Learning} en la tarea de selección de configuraciones gráficas.

\end{abstract}

%-----------------------------------------------------------------------------------
% English Abstract
%-----------------------------------------------------------------------------------
% \vspace{0.5cm}

% \begin{enabstract}

%   The English Abstract must have have $100$ to $200$ words, and present in a clear
%   and concise form the essentials of the article content.

% \end{enabstract}

%-----------------------------------------------------------------------------------
% Palabras clave
%-----------------------------------------------------------------------------------
% \begin{keywords}
% 	Separadas,
% 	Por,
% 	Comas.
% \end{keywords}

%-----------------------------------------------------------------------------------
% Temas
%-----------------------------------------------------------------------------------
% \begin{topics}
% 	Tema, Subtema.
% \end{topics}


%-----------------------------------------------------------------------------------
% NO BORRAR ESTAS LINEAS!
%-----------------------------------------------------------------------------------
\vspace{0.8cm}
]
%-----------------------------------------------------------------------------------


%===================================================================================

%===================================================================================
% Introducción
%-----------------------------------------------------------------------------------
\section{Introducción}\label{sec:intro}
%-----------------------------------------------------------------------------------
 

%===================================================================================



%===================================================================================
% Desarrollo
%-----------------------------------------------------------------------------------
\section{Desarrollo}\label{sec:dev}
%-----------------------------------------------------------------------------------
 

%-----------------------------------------------------------------------------------
	\subsection{Selecci\'on de \textit{features} de columnas}


	\subsubsection*{Medidas de dimensi\'on}
		\begin{enumerate}
			\item \textbf{Longitud}: La cantidad de elementos de una columna, esta medida obtuvo un
			alto \'indice de relevancia lo cual parece respaldar ciertas heur\'isticas 
			utilizadas de forma com\'un por analistas como pueden ser "no tener demasiadas barras en los gr\'aficos"
			o "no tener demasiadas porciones en un gr\'afico de pastel" debido a que dificultan
			la correcta observaci\'on de los datos.
		\end{enumerate} 

	\subsubsection*{Medidas de tipo}
		\begin{enumerate}
			\item \textbf{Tipo general}: El tipo general se refiere a la clasificaci\'on de la variable estad\'istica
			pudiendo ser categ\'orica (C), cuantitativa (Q) o temporal (T), esta clasificaci\'on se apoya en
			heur\'isticas comunes como utilizar variables temporales y categ\'oricas en el eje $x$.
			\item \textbf{Tipo Espec\'ifico}: Se refiere al tipo de dato utilizado para representar la variable pudiendo
			ser una cadena de texto (\textit{string}), un valor booleano (\textit{boolean}), un entero (\textit{integer}), 
			un decimal (\textit{decimal}) o una fecha (\textit{datetime}).
		\end{enumerate}
	
	\subsubsection*{Medidas de valores}

			Estas medidas se encargan de describir caracter\'isticas de los datos de la columna 
			y debido a que existen distintos tipos de variables estas medidas son dependientes del tipo.
			\begin{enumerate}
				\item Estad\'isticas [Q,T]:
					\begin{enumerate}
						% \item \textbf{Medidas de tendencia central}: Se consideraron la media, mediana, moda y la normalizaci\'on de estas medidas
						% utilizando el m\'aximo de la columna. El modelo propuesto en \cite{hu2019vizml} asignaba una baja
						% relevancia a estas medidas excepto la media normalizada. Una de las mayores desventajas que podemos
						% se\~nalar es que son muy sensibles a la magnitud de los datos lo cual dificultar\'ia la comparaci\'on de
						% datasets con rangos de valores distintos, en particular la media es especialmente sensible a la
						% existencia de valores extremos y el proceso de normalizar utilizando el m\'aximo puede agravar este hecho debido
						% a que valores extremos muy altos resultar\'ian en valores de las medidas normalizadas cercanos a 0 independientemente de la distribuci\'on.
						% Si bien existen relaciones entre estas medidas y las caracter\'isticas de la curva entre los puntos de datos
						% la existencia de algoritmos para identificar estas relaciones espec\'ificas libera al modelo de la necesidad de inferirlas a 
						% partir de medidas b\'asicas, un ejemplo de esto lo podemos observar en la relaci\'on con la asimetr\'ia \cite{mann2007introductory_asymetric} y
						% la existencia de m\'ultiples algoritmos para su c\'omputo \cite{doane2011measuring}.
						% \item Medidas de dispersi\'on: rango, rango normalizado, varianza, desviaci\'on est\'andar, coeficiente de variaci\'on, desviaci\'on
						% mediana absoluta y desviaci\'on media absoluta.
						% \item Medidas de posici\'on: $Q_1$ y $Q_3$ 
						\item \textbf{Coeficiente de variaci\'on}: El coeficiente de variaci\'on expresa la raz\'on entre la desviaci\'on t\'ipica
						y la media aritm\'etica: $$CV = \frac{s}{\overline{x}}$$ Tiene la ventaja de ser una medida de variabilidad relativa permitiendo poder
						comparar columnas con diferentes rangos de valores y unidades de medidas \cite{mann2007introductory_cv}, adem\'as su interpretaci\'on como forma de evaluar
						la homogeneidad/heterogeneidad de un conjunto de datos permite discretizar esta variable.\\\\
						\begin{tabular}{| l | l |}
							\hline
							$CV$ & \text{Interpretaci\'on}\\ \hline
							$CV \geq 0.26$ & \text{Muy heterog\'eneo}\\
							$0.16 \leq CV < 0.26$ & \text{Heterog\'eneo}\\
							$0.11 \leq CV < 0.16$ & \text{Homog\'eneo}\\
							$0 \leq CV < 0.11$ & \text{Muy homog\'eneo} \\ \hline 
						\end{tabular}
						
						\item \textbf{Coeficiente de dispersi\'on cuartil}: Este se define como:
						$$\frac {Q_3 - Q_1}{Q_3 + Q_1}$$ 
						donde $Q_1$ y $Q_3$ son el primer y tercer cuartil respectivamente. Esta medida permite comparar
						los rangos de distintos conjuntos de datos aunque tambi\'en es importante notar que es sensible a la presencia de valores extremos.
						
					\end{enumerate}

				\item Distribuci\'on [Q]:
					\begin{enumerate}
						\item \textbf{Entrop\'ia}: Se refiere a la definici\'on de \textit{entrop\'ia} en el campo de teor\'ia de la informaci\'on. Esta medida
			 			fue utilizada en \cite{seo2004rank} para establecer un \textit{ranking} entre histogramas de acuerdo a la uniformidad de la distribuci\'on
						 utilizando la siguiente definici\'on. Sea un histograma de $k$ intervalos entonces la entrop\'ia del histograma $h$ es
						 $$H(h) = -\sum_{i=1}^{k}{p_i \log_2{p_i}}$$ donde $p_i$ es la probabilidad de que un elemento pertenezca al $i$-\'esimo intervalo. 
						 Un alto valor de entrop\'ia se asocia a que los elementos pertenecen a una distribuci\'on uniforme y que el histograma tiende a ser plano.
						
						 \item \textbf{Gini}: El coeficiente de Gini es una medida de dispersi\'on definida como la media de las diferencias absolutas entre todos los
						posibles pares de individuos de una poblaci\'on para una medida dada.
						$$
							G = \frac{ \sum_{i=1}^{n} \sum_{j=1}^n {| x_i - x_j |} } {2n^2\overline{x}}
						$$
						Donde $n$ es la cantidad de medidas y $\overline{x}$ es la media aritm\'etica.
						El valor m\'inimo es 0 cuando todas las medidas son iguales, esto puede ser utilizado para medir la
						homogeneidad/heterogeneidad de los datos.
						
						\item \textbf{Skewness}: Esta medida describe la asimetr\'ia de la distribuci\'on de acuerdo a la media,
						indicando la direcci\'on y la magnitud relativa de la desviaci\'on de la distribuci\'on tomando como referencia una
						distribuci\'on normal. Es el tercer momento est\'andar definido como:
						$$
							\tilde{\mu}_3 = E \left[\left( \frac{X - \mu}{\sigma} \right)^3\right]
						$$
						
						\item \textbf{Curtosis}: Esta medida permite describir el comportamiento de los datos en la cola de la distribuci\'on, esto
						permite obtener una medida de que tan susceptible es la distribuci\'on a la aparici\'on de valores extremales. Es el cuarto momento
						est\'andar definido como:
						$$
							\tilde{\mu}_4 = E \left[\left( \frac{X - \mu}{\sigma} \right)^4\right]
						$$

						Adem\'as existe la definici\'on alternativa de exceso de curtosis la cual permite discretizar esta variable, esta se 
						define como $\tilde{\mu}_4' = \tilde{\mu}_4 - 3$ y define las clases siguientes:

						\begin{tabular}{| l | l |}
							\hline
							$\tilde{\mu}_4'$ & \text{Clase}\\ \hline
							$\tilde{\mu}_4' = 0$ & \text{Mesokurtic}\\
							$\tilde{\mu}_4' > 0$ & \text{Leptokurtic}\\
							$\tilde{\mu}_4' < 0$ & \text{Platykurtic} \\ \hline 
						\end{tabular}

						\item \textbf{Normalidad}: 
						
						\item \textbf{Momentos de orden superior}:
					\end{enumerate}
			\end{enumerate}
	 
% %-----------------------------------------------------------------------------------
	 
%-----------------------------------------------------------------------------------
% 	\subsection{Listas y Descripciones}\label{sub:lists}
% %-----------------------------------------------------------------------------------
% 		Para producir listas enumeradas, use el siguiente estilo:

% %-----------------------------------------------------------------------------------
% 		\begin{enumerate}
% 			\item Primer Elemento
% 			\item Segundo Elemento
% 			%
% 			\begin {enumerate}
% 				\item {Segundo Elemento - Subitem Uno}
% 				\item {Segundo Elemento - Subitem Dos}
% 			\end {enumerate}
% 			%
% 		\end{enumerate}

% %-----------------------------------------------------------------------------------
% 		Para producir descripciones, use el siguiente estilo:

% %-----------------------------------------------------------------------------------
% 		\begin{description}
% 			\item [Primer Elemento] con su respectiva descripción.
% 			\item [Segundo Elemento] también con su respectiva descripción.
% 		\end{description}

%-----------------------------------------------------------------------------------
% 	\subsection{Figuras}\label{sub:figures}
% %-----------------------------------------------------------------------------------
% 		Para producir cuerpos flotantes (figuras ó tablas), asegúrese de numerar
% 		y etiquetar correctamente cada figura. Las referencias a las figuras deben
% 		estar también correctamente etiquetadas. Por ejemplo, en la Fig. \ref{fig:ex}
% 		se muestra\ldots.

% 		\begin{figure}[htb]%
% 		\begin{center}
% 			\emph{Aquí va el contenido de la figura \ldots}
% 		\end{center}
% 		\caption{Figura de ejemplo \label{fig:ex}}%
% 		\end{figure}

%-----------------------------------------------------------------------------------
% 	\subsection{Código Fuente}\label{sub:listings}
% %-----------------------------------------------------------------------------------
% 		Para producir código fuente, envuélvalo en una figura flotante y
% 		etiquételo correctamente. Por ejemplo, en la Fig. \ref{fig:code}
% 		se muestra un código bastante conocido\ldots.

% 		% Configuración de Listings
% 		\lstset{keywordstyle=\color{blue}, basicstyle=\small}

% 		\begin{figure}[htb]%
% 			\begin{lstlisting}[language=c]%

%     int main(int argc, char** argv)
%     {
%         // Imprimiendo "Hola Mundo".
%         printf("Hello, World");
%     }

% 			\end{lstlisting}
% 		\caption{Código fuente de ejemplo.\label{fig:code}}
% 		\end{figure}

%-----------------------------------------------------------------------------------
	% \subsection{Referencias}
%-----------------------------------------------------------------------------------
  	% Las referencias deben estar agrupadas en una sección al final del artículo,
  	% y las citas numeradas correctamente, por ejemplo \cite{knuth} ó \cite{goedel}.

  	% Incluya toda la información importante de cada referencia, incluídos autor,
  	% título, y notas de la edición. En caso de citar sitios web, además
  	% de la URL, incluya la fecha en que fue consultado, como en \cite{wiki}.

%===================================================================================



%===================================================================================
% Conclusiones
%-----------------------------------------------------------------------------------
\section{Conclusiones}\label{sec:conc}

%   En esta sección puede incluir las conclusiones de su investigación y las ideas
%   sobre la continuidad del trabajo, en el caso que aplique.

%===================================================================================



%===================================================================================
% Recomendaciones
%-----------------------------------------------------------------------------------
\section{Recomendaciones}\label{sec:rec}

%   En esta sección puede incluir recomendaciones sobre posibles formas de continuar
%   la investigación u otros temas relacionados.

%===================================================================================



%===================================================================================
% Bibliografía
%-----------------------------------------------------------------------------------
% \begin{thebibliography}{99}
%-----------------------------------------------------------------------------------
	
\bibliography{biblio}
% \bibitem{knuth} Donald E. Knuth. \emph{The Art of Computer Programming}.
	% 	Volume 1: Fundamental Algorithms (3rd~edition), 1997.
	% 	Addison-Wesley Professional.

	% \bibitem{goedel} Kurt Göedel. \emph{Über formal unentscheidbare Sätze der
	% 	Principia Mathematica und verwandter Systeme, I}.
	% 	Monatshefte für Mathematik und Physik 38.

	% \bibitem{wiki} Wikipedia. URL: \href{http://en.wikipedia.org}
	%   {http://en.wikipedia.org}.
	% 	Consultado en \today.

%-----------------------------------------------------------------------------------
% \end{thebibliography}

%-----------------------------------------------------------------------------------

\label{end}

\end{document}

%===================================================================================
